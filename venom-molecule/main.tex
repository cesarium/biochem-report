\documentclass[12pt,a4paper]{article}
\usepackage[utf8]{inputenc}
\usepackage[portuguese]{babel}
\usepackage[margin=2cm]{geometry}
%\setlength{\parindent}{13pt}
\usepackage{listings,mathptmx,amsfonts,amsmath,amssymb,amsthm}
\linespread{1.3}% equals 1.5 of line spacing

\usepackage[font=footnotesize,labelfont=bf,tableposition=top]{caption}

%\usepackage[backend=bibtex,citestyle=numeric-comp,bibstyle=ieee,sorting=none,minbibnames=5,maxbibnames=5,defernumbers=true,firstinits=true]{biblatex}

\usepackage{subcaption,mathtools,float,cite,lipsum,bm,physics,pythontex,caption}
%\usepackage[colorlinks]{hyperref}
\usepackage{sectsty,isotope,xcolor}
\usepackage{graphicx}

%\hypersetup{linkcolor=blue}
%\hypersetup{citecolor=blue}

\usepackage[colorlinks = true,
linkcolor = blue,
urlcolor  = blue,
citecolor = blue,
anchorcolor = blue]{hyperref}

\usepackage[nameinlink,capitalize]{cleveref}
\usepackage[bottom]{footmisc}

\title{\textbf{Molécula Magnífica: Veneno\\de Abelha - Phospholipase A2}}
\author{\textbf{António Cesário}\\\\%
	Faculdade de Ciências da Universidade do Porto\\%
	Mestrado em Física Experimental\\%
	Bioquímica Computacional%\\%
}
\date{\today}

\begin{document}
\maketitle
\noindent\rule{\textwidth}{0.4pt}
\paragraph{}

\begin{flushright}\textit{``O que arde cura (...)''} - Ditado Popular Português\end{flushright}
%%%%%%%%%%%%%%%%%%%%%%%%%%%%%%%%%%%%%%%%%%%%%%%%%%%%%

O veneno da abelha do mel (\textit{Apis mellifera}) contém uma larga gama de proteínas e toxinas \cite{HoneybeeVenomRich,sonTherapeuticApplicationAntiarthritis2007}. Algumas destas toxinas têm muito interesse para a farmacologia, sendo usadas há muitos anos para tratar doenças como atrites, e estão atualmente a ser investigados pelas suas propriedades radio-protetoras, antimutagénicas, anti-cancerígenas, entre outras \cite{leeBeeVenomPhospholipase2016}.

O composto \textit{Phospholipase A2 de Grupo III} (III sPLA$_2$) constitui cerca de 10\% do veneno seco e é conhecido por ser o alergénio principal destas abelhas, podendo causar choques anafiláticos. Contudo, o III sPLA$_2$ revela propriedades muito interessantes e benéficas para o ser humano, tendo sido estudada a sua integração para o tratamento em doenças como asma, Parkinson, e outros efeitos anti-inflamatórios \cite{leeBeeVenomPhospholipase2016}. Porém, é necessário encontrar um compromisso entre os perigos e os benefícios deste composto.

A estrutura molecular do III sPLA$_2$ pode ser encontrada em \href{https://www.rcsb.org/structure/1POC}{DOI:10.2210/pdb1POC/pdb}. Para concluir o desafio da molécula magnífica, remeto para a imagem da \cref{fig:}, onde está retratada a molécula do composto mencionado. A estrutura foi representada através das funcionalidades do site \href{https://www.rcsb.org/3d-view/ngl/1poc}{rcsb.org} usando o NGL como \textit{viewer}.

\begin{figure}[H]
	\centering
	\includegraphics[width=\textwidth]{latex-image.jpg}
	\caption{Representação da estrutura do composto III sPLA$_2$ no abdómen de uma abelha. A imagem original pode ser encontrada \href{https://post.healthline.com/wp-content/uploads/2020/08/732x549_How_Is_An_Infected_Bee_Sting_Treated-1-732x549.jpg}{neste link}. Edição de imagem: José Cesário.}
	\label{fig:}
\end{figure} 


%%%%%%%%%%%%%%%%%%%%%%%%%%%%%%%%%%%%%%%%%%%%%%%%%%%%%
\paragraph{Agradecimentos:} Agradeço ao meu irmão, José Cesário, por ter editado a imagem.
%%%%%%%%%%%%%%%%%%%%%%%%%%%%%%%%%%%%%%%%%%%%%%%%%%%%%
\bibliographystyle{unsrt}
\bibliography{biblio}
\end{document}